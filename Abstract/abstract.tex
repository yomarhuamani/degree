\begin{titlepage}
	\title{Abstract}
	
	 \begin{abstract}

		\linespread{3}Tesla Technologies - Perú, es una empresa dedicada a aportar e \linespread{1.5}implementar soluciones en materia de Riesgos operativos, \linespread{1.5}Seguridad de la información, Continuidad del negocio, Auditoría \linespread{1.5}entre otros, con sede en Lima Perú y una sucursal en la ciudad de Santiago de Chile.\\
		
		En el camino de atender las necesidades de los clientes y apoyarlos en la creación de informes gerenciales Tesla Technologies decide integrar funcionalidades dinámicas de otros proyectos ya consolidados para lograr crear una ventaja competitiva.
		Se inició con la búsqueda de productos de software que permitieran crear reportes de forma dinámica y puedan integrarse a los productos ofrecidos por Tesla Technologies, se revisaron algunas herramientas del cuadrante de Gartner y del cuadrante Forrester siendo la mayoria de estas rechazadas por el alto costo de sus licencias e implementación. 
		Como resultado de la revisión a estas herramientas, fue el proyecto Saiku Analytics complemento de la suite Pentaho BI  que fue seleccionado por su facilidad y compatibilidad con los productos ofrecidos por Tesla Technologies.
		Se revisaron las licencias usadas en el proyecto Saiku Analytics y el impacto legal generado en la integración con los productos de Tesla Technologies, se establecieron las funcionalidades a conservar y las nuevas a desarrollarse alineadas a cumplir con los objetivos establecidos.\\
		
		Se analizaron las tecnologías, librerias de desarrollo y el impacto a generar en los productos ofrecidos por Tesla Technologies con la integración del proyecto Saiku Analytics.
		Se procedió con la implementación de la integración del proyecto Saiku Analytics y sus modificaciones que luego de la integración fue llamado proyecto Rubik Report.
		Finalmente luego del lanzamiento del módulo Rubik Report se procedió a crear y diseñar reportes dinámicos y se evaluó los beneficios como resultado de la integración con los productos ofrecidos por Tesla Technologies.\\
	\end{abstract}
	\maketitle
\end{titlepage}
