\chapter{Marco Teórico} \label{chapter:II}
\section{Open Source}
		\lipsum[1-2]
%------------------------------------------------------------------------------------------------------------------
%------------------------------------- Saiku Analytics ------------------------------------------------------------
%------------------------------------------------------------------------------------------------------------------	
\section{Saiku Analytics}
	\subsection{Plugin}
			\lipsum[1-2]
	\subsection{Pentago BI}
			\lipsum[1-2]
	\subsection{Cuadrante M\'{a}gico de Gartner}
			\lipsum[1-2]
	\subsection{Licencias}
			\lipsum[1-2]
		\subsubsection{Licencia Apache}
				\lipsum[1-2]
		\subsubsection{Licencia GPL}
				\lipsum[1-2]
		\subsubsection{Licencia LGPL}
				\lipsum[1-2]
		\subsubsection{Licencia MIT}
				\lipsum[1-2]
%------------------------------------------------------------------------------------------------------------------
%-------------------------------------  SCRUM  --------------------------------------------------------------------
%------------------------------------------------------------------------------------------------------------------
\section{SCRUM}
	\subsection{Definici\'{o}n de Scrum}
	\subsection{Teor\'{i}a de Scrum}
		\subsubsection{Transparencia}
		\subsubsection{Inspecci\'{o}n}
		\subsubsection{Adaptaci\'{o}n}
	\subsection{Valores de Scrum}
	\subsection{\textit{Scrum Team}}
		\subsubsection{Developers}
		\subsubsection{Product Owner}
		\subsubsection{Scrum Master}
	\subsection{Eventos de Scrum}
		\subsubsection{El \textit{Sprint}}
		\subsubsection{\textit{Sprint Planning}}
		\subsubsection{\textit{Daily Scrum}}
		\subsubsection{\textit{Sprint Review}}
		\subsubsection{\textit{Sprint Retrospective}}
	\subsection{Artefactos de Scrum}
		\subsubsection{\textit{Product Backlog}}
		\subsubsection{\textit{Sprint Backlog}}
		\subsubsection{\textit{Increment}}
	\subsection{Cambios de la gu\'{i}a Scrum 2017 a la gu\'{i}a Scrum 2020}
%------------------------------------------------------------------------------------------------------------------	
%-----------------------------  Extreme Programming ---------------------------------------------------------------
%------------------------------------------------------------------------------------------------------------------
\section{Programaci\'{o}n Extrema(XP)}
		\lipsum[1-2]
	\subsection{Valores}
	\subsection{Principios}
	\subsection{Pr\'{a}cticas}
		\subsubsection{Pr\'{a}cticas primarias}
		\subsubsection{Pr\'{a}cticas corolarias}
	\subsection{Planeaci\'{o}n}
	\subsection{Diseno}
	\subsection{Pruebas}
%------------------------------------------------------------------------------------------------------------------
%-------------------------------Tecnologias backend para la integracion--------------------------------------------
%------------------------------------------------------------------------------------------------------------------
\section{Tecnolog\'{i}as BackEnd para la integraci\'{o} del proyecto Saiku Analytics}
		\lipsum[1-2]
	\subsection{Lenguaje de programaci\'{o}n Java}
			\lipsum[1-2]
	\subsection{Java Specification Requests(JSRs)}
		\subsubsection{JSR-366}
				\lipsum[1-2]
		\subsubsection{JSR-47}
				\lipsum[1-2]
		\subsubsection{JSR-338}
				\lipsum[1-2]
		\subsubsection{JSR-346}
				\lipsum[1-2]
		\subsubsection{JSR-369}
				\lipsum[1-2]
		\subsubsection{JSR-370}
		\lipsum[1-2]
		\subsubsection{JSR-371}
		\lipsum[1-2]
	\subsection{\textit{Spring Framework}}
			\lipsum[1-2]
		\subsubsection{\textit{Spring security}}
				\lipsum[1-2]
	\subsection{Maven}
			\lipsum[1-2]
	\subsection{Mondian}
			\lipsum[1-2]
	\subsection{Git}
			\lipsum[1-2]
	\subsection{Enunciate}
			\lipsum[1-2]
%------------------------------------------------------------------------------------------------------------------
%-------------------------------Tecnologias backend para la integracion--------------------------------------------
%------------------------------------------------------------------------------------------------------------------
\section{Tecnolog\'{i}as FrontEnd para la integraci\'{o}n
		 del proyecto Saiku Analytics}
		\lipsum[1-2]
	\subsection{ECMAScript6 (ES6)}
			\lipsum[1-2]
	\subsection{HTML5}
			\lipsum[1-2]
	\subsection{Javascript}
			\lipsum[1-2]
	\subsection{Css}
			\lipsum[1-2]
	\subsection{Backbone}
			\lipsum[1-2]
	\subsection{Node.js}
			\lipsum[1-2]
	\subsection{CCC-Charts}
			\lipsum[1-2]

		
		
		
		
%JCache (JSR-107)
%Java API for JSON Binding (JSR-367)
%Model View Controller (MVC) (JSR-371)
%Java API for WebSocket
%Java API for JSON Processing (JSON-P)
%Java API for RESTful Web Services (JAX-RS) JSR-370
%JavaServer Faces (JSF)
%Java Servlet JSR-369
%Expression Language (EL)
%Interceptors
%Java Message Service (JMS)
%Concurrency Utilities for Java EE
%Batch Applications for the Java Platform
%Contexts and Dependency Injection for Java EE (CDI) JSR-346
%Bean Validation
%Common Annotations
%Java Connector Architecture
%Java Transaction API (JTA)
%Java Persistence API (JPA) JSR-338
%Enterprise JavaBeans (EJB)
%JavaServer Pages (JSP)
%Web-Profile