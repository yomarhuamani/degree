\chapter{Marco Teórico} \label{chapter:II}
\begin{spacing}{1.5}
\section{Open Source}
		Traducito literalemte como "fuente abierta", refiri\'{e}ndose as\'{i} a una caracter\'{i}stica del software donde el c\'{o}digo fuente es de libre acceso, sin embargo, Open Source no solo significa eso. Los t\'{e}rminos de distribuci\'{o}n de un software Open Source deben cumplir con los siguientes criterios:

		\begin{enumerate}[label=\alph*)]
			\item \textbf{Distribuci\'{o}n gratuita}\\
			La licencia no impedirá a ninguna parte vender o regalar el software como un componente de una distribución de software agregada que contenga programas de varias fuentes diferentes. La licencia no requerirá regalías u otros honorarios por dicha venta \cite{chap2_cite_1}.
			
			\item \textbf{C\'{o}digo fuente}\\
			El programa debe incluir el código fuente y debe permitir la distribución en código fuente y en forma compilada. Cuando alguna forma de un producto no se distribuye con el código fuente, debe haber un medio que est\'{e} bien publicitado para obtener el código fuente por un costo de reproducción no superior al razonable, preferiblemente, descargarlo a través de Internet sin cargo. El código fuente debe ser la forma preferida en la que un programador modificaría el programa. No se permite el código fuente deliberadamente ofuscado. No se permiten formas intermedias como la salida de un preprocesador o traductor \cite{chap2_cite_1}.

			\item \textbf{Trabajos derivados}\\
			La licencia debe permitir modificaciones y trabajos derivados, y debe permitir que se distribuyan en los mismos términos que la licencia del software original \cite{chap2_cite_1}.
			
			\item \textbf{Integridad del autor del c\'{o}digo fuente}\\
			La licencia puede restringir la distribución del código fuente en forma modificada solo si la licencia permite la distribución de "archivos de parche" con el código fuente con el fin de modificar el programa en el momento de la compilación. La licencia debe permitir explícitamente la distribución de software creado a partir del código fuente modificado. La licencia puede requerir que las obras derivadas lleven un nombre o número de versión diferente al del software original \cite{chap2_cite_1}.
			
			\item \textbf{No discriminaci\'{o}n contra persona y grupos}\\
			La licencia no debe discriminar a ninguna persona o grupo de personas \cite{chap2_cite_1}.
			
			\item \textbf{No discrininaci\'{o} en los lugares de actividad}\\
			La licencia no debe restringir a nadie el uso del programa en un campo específico de actividad. Por ejemplo, no puede restringir el uso del programa en una empresa o para la investigación genética \cite{chap2_cite_1}.
			
			\item \textbf{Licencia de distribuci\'{o}n}\\
			Los derechos adjuntos al programa deben aplicarse a todos aquellos a quienes se redistribuye el programa sin la necesidad de que esas partes ejecuten una licencia adicional \cite{chap2_cite_1}.
			
			\item \textbf{La licencia no debe ser espec\'{i}fica para un producto}\\
			Los derechos adjuntos al programa no deben depender de que el programa sea parte de una distribución de software en particular. Si el programa se extrae de esa distribución y se usa o distribuye dentro de los términos de la licencia del programa, todas las partes a quienes se redistribuye el programa deben tener los mismos derechos que los que se otorgan junto con la distribución original del software \cite{chap2_cite_1}.
			
			\item \textbf{La licencia no debe restringir a otro software}\\
			
		\end{enumerate}
%------------------------------------------------------------------------------------------------------------------
%------------------------------------- Saiku Analytics ------------------------------------------------------------
%------------------------------------------------------------------------------------------------------------------	
\section{Saiku Analytics}
	\subsection{Plugin}
			\lipsum[1-2]
	\subsection{Pentago BI}
			\lipsum[1-2]
	\subsection{Cuadrante M\'{a}gico de Gartner}
			\lipsum[1-2]
	\subsection{Licencias}
			\lipsum[1-2]
		\subsubsection{Licencia Apache}
				\lipsum[1-2]
		\subsubsection{Licencia GPL}
				\lipsum[1-2]
		\subsubsection{Licencia LGPL}
				\lipsum[1-2]
		\subsubsection{Licencia MIT}
				\lipsum[1-2]
%------------------------------------------------------------------------------------------------------------------
%-------------------------------------  SCRUM  --------------------------------------------------------------------
%------------------------------------------------------------------------------------------------------------------
\section{SCRUM}
			\lipsum[1]
	\subsection{Definii\'{o}n de Scrum}
				\lipsum[1]
	\subsection{Teor\'{i}a de Scrum}
				\lipsum[1]
		\subsubsection{Transparencia}
					\lipsum[1]
		\subsubsection{Inspecci\'{o}n}
					\lipsum[1]
		\subsubsection{Adaptaci\'{o}n}
					\lipsum[1]
	\subsection{Valores de Scrum}
				\lipsum[1]
	\subsection{\textit{Scrum Team}}
				\lipsum[1]
		\subsubsection{Developers}
					\lipsum[1]
		\subsubsection{Product Owner}
					\lipsum[1]
		\subsubsection{Scrum Master}
					\lipsum[1]
	\subsection{Eventos de Scrum}
				\lipsum[1]
		\subsubsection{El \textit{Sprint}}
					\lipsum[1]
		\subsubsection{\textit{Sprint Planning}}
					\lipsum[1]
		\subsubsection{\textit{Daily Scrum}}
					\lipsum[1]
		\subsubsection{\textit{Sprint Review}}
					\lipsum[1]
		\subsubsection{\textit{Sprint Retrospective}}
					\lipsum[1]
	\subsection{Artefactos de Scrum}
				\lipsum[1]
		\subsubsection{\textit{Product Backlog}}
					\lipsum[1]
		\subsubsection{\textit{Sprint Backlog}}
					\lipsum[1]
		\subsubsection{\textit{Increment}}
					\lipsum[1]
	\subsection{Cambios de la gu\'{i}a Scrum 2017 a la gu\'{i}a Scrum 2020}
				\lipsum[1]
%------------------------------------------------------------------------------------------------------------------	
%-----------------------------  Extreme Programming ---------------------------------------------------------------
%------------------------------------------------------------------------------------------------------------------
\section{Programaci\'{o}n Extrema(XP)}
		\lipsum[1-2]
	\subsection{Valores, Principios y Pr\'{a}cticas}
			\lipsum[1]
	\subsection{Valores}
				\lipsum[1]
	\subsection{Principios}
				\lipsum[1]
	\subsection{Pr\'{a}cticas}
				\lipsum[1]
		\subsubsection{Pr\'{a}cticas primarias}
					\lipsum[1]
		\subsubsection{Pr\'{a}cticas corolarias}
					\lipsum[1]
	\subsection{Planeaci\'{o}n}
				\lipsum[1]
	\subsection{Dise\'{n}o}
				\lipsum[1]
	\subsection{Pruebas}
				\lipsum[1]
%------------------------------------------------------------------------------------------------------------------
%-------------------------------Tecnologias backend para la integracion--------------------------------------------
%------------------------------------------------------------------------------------------------------------------
\section{Tecnolog\'{i}as BackEnd para la integraci\'{o} del proyecto Saiku Analytics}
		\lipsum[1-2]
	\subsection{Lenguaje de programaci\'{o}n Java}
			\lipsum[1-2]
	\subsection{Java Specification Requests(JSRs)}
		\subsubsection{JSR-366}
				\lipsum[1-2]
		\subsubsection{JSR-47}
				\lipsum[1-2]
		\subsubsection{JSR-338}
				\lipsum[1-2]
		\subsubsection{JSR-346}
				\lipsum[1-2]
		\subsubsection{JSR-369}
				\lipsum[1-2]
		\subsubsection{JSR-370}
		\lipsum[1-2]
		\subsubsection{JSR-371}
		\lipsum[1-2]
	\subsection{\textit{Spring Framework}}
			\lipsum[1-2]
		\subsubsection{\textit{Spring security}}
				\lipsum[1-2]
	\subsection{Maven}
			\lipsum[1-2]
	\subsection{Mondian}
			\lipsum[1-2]
	\subsection{Git}
			\lipsum[1-2]
	\subsection{Enunciate}
			\lipsum[1-2]
%------------------------------------------------------------------------------------------------------------------
%-------------------------------Tecnologias backend para la integracion--------------------------------------------
%------------------------------------------------------------------------------------------------------------------
\section{Tecnolog\'{i}as FrontEnd para la integraci\'{o}n
		 del proyecto Saiku Analytics}
		\lipsum[1-2]
	\subsection{ECMAScript6 (ES6)}
			\lipsum[1-2]
	\subsection{HTML5}
			\lipsum[1-2]
	\subsection{Javascript}
			\lipsum[1-2]
	\subsection{Css}
			\lipsum[1-2]
	\subsection{Backbone}
			\lipsum[1-2]
	\subsection{Node.js}
			\lipsum[1-2]
	\subsection{CCC-Charts}
			\lipsum[1-2]
\end{spacing}
		
		
		
%JCache (JSR-107)
%Java API for JSON Binding (JSR-367)
%Model View Controller (MVC) (JSR-371)
%Java API for WebSocket
%Java API for JSON Processing (JSON-P)
%Java API for RESTful Web Services (JAX-RS) JSR-370
%JavaServer Faces (JSF)
%Java Servlet JSR-369
%Expression Language (EL)
%Interceptors
%Java Message Service (JMS)
%Concurrency Utilities for Java EE
%Batch Applications for the Java Platform
%Contexts and Dependency Injection for Java EE (CDI) JSR-346
%Bean Validation
%Common Annotations
%Java Connector Architecture
%Java Transaction API (JTA)
%Java Persistence API (JPA) JSR-338
%Enterprise JavaBeans (EJB)
%JavaServer Pages (JSP)
%Web-Profile