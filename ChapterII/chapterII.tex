\chapter{Marco Teórico} \label{chapter:II}
\begin{spacing}{1.5}
\section{Open Source}
		Traducito literalemte como "fuente abierta", refiri\'{e}ndose as\'{i} a una caracter\'{i}stica del software donde el c\'{o}digo fuente es de libre acceso, sin embargo, Open Source no solo significa eso. Los t\'{e}rminos de distribuci\'{o}n de un software Open Source deben cumplir con los siguientes criterios:

		\begin{enumerate}
			\item \textbf{Distribuci\'{o}n gratuita}\\
			La licencia no impedirá a ninguna parte vender o regalar el software como un componente de una distribución de software agregada que contenga programas de varias fuentes diferentes. La licencia no requerirá regalías u otros honorarios por dicha venta \cite{chap2_open_source}.
			
			\item \textbf{C\'{o}digo fuente}\\
			El programa debe incluir el código fuente y debe permitir la distribución en código fuente y en forma compilada. Cuando alguna forma de un producto no se distribuye con el código fuente, debe haber un medio que est\'{e} bien publicitado para obtener el código fuente por un costo de reproducción no superior al razonable, preferiblemente, descargarlo a través de Internet sin cargo. El código fuente debe ser la forma preferida en la que un programador modificaría el programa. No se permite el código fuente deliberadamente ofuscado. No se permiten formas intermedias como la salida de un preprocesador o traductor \cite{chap2_open_source}.

			\item \textbf{Trabajos derivados}\\
			La licencia debe permitir modificaciones y trabajos derivados, y debe permitir que se distribuyan en los mismos términos que la licencia del software original \cite{chap2_open_source}.
			
			\item \textbf{Integridad del autor del c\'{o}digo fuente}\\
			La licencia puede restringir la distribución del código fuente en forma modificada solo si la licencia permite la distribución de "archivos de parche" con el código fuente con el fin de modificar el programa en el momento de la compilación. La licencia debe permitir explícitamente la distribución de software creado a partir del código fuente modificado. La licencia puede requerir que las obras derivadas lleven un nombre o número de versión diferente al del software original \cite{chap2_open_source}.
			
			\item \textbf{No discriminaci\'{o}n contra persona y grupos}\\
			La licencia no debe discriminar a ninguna persona o grupo de personas \cite{chap2_open_source}.
			
			\item \textbf{No discrininaci\'{o}n en los lugares de actividad}\\
			La licencia no debe restringir a nadie el uso del programa en un campo específico de actividad. Por ejemplo, no puede restringir el uso del programa en una empresa o para la investigación genética \cite{chap2_open_source}.
			
			\item \textbf{Licencia de distribuci\'{o}n}\\
			Los derechos adjuntos al programa deben aplicarse a todos aquellos a quienes se redistribuye el programa sin la necesidad de que esas partes ejecuten una licencia adicional \cite{chap2_open_source}.
			
			\item \textbf{La licencia no debe ser espec\'{i}fica para un producto}\\
			Los derechos adjuntos al programa no deben depender de que el programa sea parte de una distribución de software en particular. Si el programa se extrae de esa distribución y se usa o distribuye dentro de los términos de la licencia del programa, todas las partes a quienes se redistribuye el programa deben tener los mismos derechos que los que se otorgan junto con la distribución original del software \cite{chap2_open_source}.
			
			\item \textbf{La licencia no debe restringir a otro software}\\
			La licencia no debe imponer restricciones a otro software que se distribuya junto con el software con licencia. Por ejemplo, la licencia no debe insistir en que todos los demás programas distribuidos en el mismo medio deben ser software de código abierto \cite{chap2_open_source}.
			
			\item \textbf{La licencia debe ser de tecnología neutral}\\
			Ninguna disposición de la licencia puede basarse en una tecnología individual o estilo de interfaz \cite{chap2_open_source}.
			
		\end{enumerate}
	
\section{Legalidad del software}
	\subsection{Licencia}
	Una licencia es la autorización que se concede para explotar con fines industriales o comerciales una patente, marca o derecho. Diccionario de la lengua española (23.ª ed.)
	
	\subsection{Derecho de autor o Copyright}
	Derecho que la ley reconoce al autor de una obra intelectual o artística para autorizar su reproducción y participar en los beneficios que esta genere. Diccionario de la lengua española (23.ª ed.)
	
	\subsection{Patente}
	Documento expedido por una autoridad en que se acredita una condición o un mérito o se da la autorización para hacer algo. Diccionario de la lengua española (23.ª ed.)
	
	\subsection{Licencia de Software}
	Una licencia de software es un contrato entre el licenciante (autor/titular de los derechos de explotación/distribución) y el licenciatario (usuario consumidor, profesional o empresa) del programa informático, para utilizarlo cumpliendo una serie de términos y condiciones establecidas dentro de sus cláusulas, es decir, es un conjunto de permisos que un licenciante otorga a un usuario en los que tiene la posibilidad de distribuir, usar o modificar el producto bajo una licencia determinada. Además se suelen definir los plazos de duración, el territorio donde se aplica la licencia (ya que la licencia se soporta en las leyes particulares de cada país o región), entre otros.
	
	\subsection{Licencia Open Source}
	Las licencias de código abierto son licencias que cumplen con la definición de código abierto; en resumen, permiten que el software se utilice, modifique y comparta libremente. Para ser aprobada por Open Source Initiative (también conocida como OSI), una licencia debe pasar por el proceso de revisión de licencias de Open Source Initiative.\cite{chap2_open_source_2}\\
	
	Existen 100 licencias registradas en la Open Source Initiative(OSI), de las cual el presente proyecto describi\'{a} 3 de ellas por ser las m\'{a}s importante del proyecto Saiku Analytics.

	\subsubsection{Licencia Apache}
	La licencia de Apache solo requiere un reconocimiento en la ``documentación del usuario final" o ``en el software mismo", no en ``todos los materiales publicitarios". La licencia de Apache no especifica la prominencia que debe darse a ese reconocimiento. La licencia de Apache es consistente con los Principios de código abierto porque no interfiere con la libertad de modificar o crear trabajos derivados de software de código abierto.\cite{chap2_apache_license} \\
	Se deben añadir dos archivos en el directorio principal de los paquetes de software redistribuidos.
	\begin{enumerate}
		\item LICENSE: Una copia de la licencia
		\item NOTICE: Un documento de texto que incluye los “avisos” obligatorios del software presente en la distribución y una copia legible de estas notificaciones deben ser distribuidas como parte de los trabajos derivados, dentro de la forma de código fuente o documentación, o dentro de una pantalla generada por las obras derivadas ( donde aparecen normalmente este tipo de notificaciones a terceros).
	\end{enumerate}			

	\subsubsection{Licencia GPL}
	Creada por la Free Software Foundation, es una licencia de software de derecho de autor ampliamente usada en el mundo del software libre y código abierto, que garantiza a los usuarios finales (personas, organizaciones o compañias), la libertad de usar, estudiar, compartir y modificar el software.\\
	Se caracteriza por el doble propósito: declarar que el software cubierto por esta licencia es libre, y protegerlo mediante el copyleft de intentos de apropiación que restrinjan esas libertades a nuevos usuarios cada vez que la obra es destruida, modificada o ampliada.

	\subsubsection{Licencia MIT}
	La licencia MIT es una licencia de software libre permisiva que se originó en el Instituto de Tecnología de Massachusetts (MIT) a fines de la década de 1980. Como licencia permisiva, solo impone restricciones muy limitadas a la reutilización y, por lo tanto, tiene una alta compatibilidad de licencias.\\
	
	La licencia MIT es compatible con muchas licencias copyleft, como la GNU General Public License (GPL); El software con licencia del MIT puede volver a licenciarse como software GPL e integrarse con otro software GPL, pero no al revés. La licencia MIT también permite la reutilización dentro del software propietario, siempre que todas las copias del software con licencia incluyan una copia de los términos de la licencia MIT y el aviso de derechos de autor, o que se vuelva a licenciar el software para eliminar este requisito. El software con licencia del MIT también se puede volver a licenciar como software propietario, lo que lo distingue de las licencias de software con copyleft. A partir de 2020, MIT fue la licencia de software más popular encontrada en un análisis, a partir de los informes de 2015 de que MIT era la licencia de software más popular en GitHub, por delante de cualquier variante de GPL y otras licencias de software libre y de código abierto (FOSS).
	
%------------------------------------------------------------------------------------------------------------------
%------------------------------------- Saiku Analytics ------------------------------------------------------------
%------------------------------------------------------------------------------------------------------------------	
\section{Saiku Analytics}
	Saiku fue fundada en 2008 por Tom Barber y Paul Stoellberger. Originalmente llamada \textit{Pentaho Analysis Tool}, comenzó como un contenedor básico basado en GWT alrededor de la biblioteca OLAP4J. A lo largo de los años ha evolucionado, y después de una reescritura completa en 2010, renació como Saiku.
	
	Saiku ofrece una solución de análisis basada en web fácil de usar que permite a los usuarios analizar rápida y fácilmente los datos corporativos y crear y compartir informes. La solución se conecta a una variedad de servidores OLAP, incluidos Mondrian, Microsoft Analysis Services, SAP BW y Oracle Hyperion y se puede implementar de manera rápida y rentable para permitir a los usuarios explorar datos en tiempo real.
	
	Saiku es una popular herramienta de análisis gráfico de código abierto para Mondrian que se puede ejecutar de forma independiente o como un complemento de Pentaho \cite{chap2_mondrian_action}.
	
	\subsection{Pentago Business Analytics}
		La plataforma Pentaho Business Analytics (BA) le permite acceder, integrar, manipular, visualizar y analizar de forma segura sus activos de big data. Ya sea que los datos estén almacenados en un archivo plano, una base de datos relacional, Hadoop, una base de datos NoSQL, una base de datos analítica, transmisiones de redes sociales, tiendas operativas o en la nube.\\
		Pentaho Business Analytics (BA) se divide en tres componentes:
		
	\subsection{Cuadrante M\'{a}gico de Gartner}
			\lipsum[1-2]

%------------------------------------------------------------------------------------------------------------------
%-------------------------------------  SCRUM  --------------------------------------------------------------------
%------------------------------------------------------------------------------------------------------------------
\section{SCRUM}
			\lipsum[1]
	\subsection{Definii\'{o}n de Scrum}
				\lipsum[1]
	\subsection{Teor\'{i}a de Scrum}
				\lipsum[1]
		\subsubsection{Transparencia}
					\lipsum[1]
		\subsubsection{Inspecci\'{o}n}
					\lipsum[1]
		\subsubsection{Adaptaci\'{o}n}
					\lipsum[1]
	\subsection{Valores de Scrum}
				\lipsum[1]
	\subsection{\textit{Scrum Team}}
				\lipsum[1]
		\subsubsection{Developers}
					\lipsum[1]
		\subsubsection{Product Owner}
					\lipsum[1]
		\subsubsection{Scrum Master}
					\lipsum[1]
	\subsection{Eventos de Scrum}
				\lipsum[1]
		\subsubsection{El \textit{Sprint}}
					\lipsum[1]
		\subsubsection{\textit{Sprint Planning}}
					\lipsum[1]
		\subsubsection{\textit{Daily Scrum}}
					\lipsum[1]
		\subsubsection{\textit{Sprint Review}}
					\lipsum[1]
		\subsubsection{\textit{Sprint Retrospective}}
					\lipsum[1]
	\subsection{Artefactos de Scrum}
				\lipsum[1]
		\subsubsection{\textit{Product Backlog}}
					\lipsum[1]
		\subsubsection{\textit{Sprint Backlog}}
					\lipsum[1]
		\subsubsection{\textit{Increment}}
					\lipsum[1]
	\subsection{Cambios de la gu\'{i}a Scrum 2017 a la gu\'{i}a Scrum 2020}
				\lipsum[1]
%------------------------------------------------------------------------------------------------------------------	
%-----------------------------  Extreme Programming ---------------------------------------------------------------
%------------------------------------------------------------------------------------------------------------------
\section{Programaci\'{o}n Extrema(XP)}
		\lipsum[1-2]
	\subsection{Valores, Principios y Pr\'{a}cticas}
			\lipsum[1]
	\subsection{Valores}
				\lipsum[1]
	\subsection{Principios}
				\lipsum[1]
	\subsection{Pr\'{a}cticas}
				\lipsum[1]
		\subsubsection{Pr\'{a}cticas primarias}
					\lipsum[1]
		\subsubsection{Pr\'{a}cticas corolarias}
					\lipsum[1]
	\subsection{Planeaci\'{o}n}
				\lipsum[1]
	\subsection{Dise\'{n}o}
				\lipsum[1]
	\subsection{Pruebas}
				\lipsum[1]
%------------------------------------------------------------------------------------------------------------------
%-------------------------------Tecnologias backend para la integracion--------------------------------------------
%------------------------------------------------------------------------------------------------------------------
\section{Tecnolog\'{i}as BackEnd para la integraci\'{o} del proyecto Saiku Analytics}
		\lipsum[1-2]
	\subsection{Lenguaje de programaci\'{o}n Java}
			\lipsum[1-2]
	\subsection{Java Specification Requests(JSRs)}
		\subsubsection{JSR-366}
				\lipsum[1-2]
		\subsubsection{JSR-47}
				\lipsum[1-2]
		\subsubsection{JSR-338}
				\lipsum[1-2]
		\subsubsection{JSR-346}
				\lipsum[1-2]
		\subsubsection{JSR-369}
				\lipsum[1-2]
		\subsubsection{JSR-370}
		\lipsum[1-2]
		\subsubsection{JSR-371}
		\lipsum[1-2]
	\subsection{\textit{Spring Framework}}
			\lipsum[1-2]
		\subsubsection{\textit{Spring security}}
				\lipsum[1-2]
	\subsection{Maven}
			\lipsum[1-2]
	\subsection{Mondian}
			\lipsum[1-2]
	\subsection{Git}
			\lipsum[1-2]
	\subsection{Enunciate}
			\lipsum[1-2]
%------------------------------------------------------------------------------------------------------------------
%-------------------------------Tecnologias backend para la integracion--------------------------------------------
%------------------------------------------------------------------------------------------------------------------
\section{Tecnolog\'{i}as FrontEnd para la integraci\'{o}n
		 del proyecto Saiku Analytics}
		\lipsum[1-2]
	\subsection{ECMAScript6 (ES6)}
			\lipsum[1-2]
	\subsection{HTML5}
			\lipsum[1-2]
	\subsection{Javascript}
			\lipsum[1-2]
	\subsection{Css}
			\lipsum[1-2]
	\subsection{Backbone}
			\lipsum[1-2]
	\subsection{Node.js}
			\lipsum[1-2]
	\subsection{CCC-Charts}
			\lipsum[1-2]
\end{spacing}
		
		
		
%JCache (JSR-107)
%Java API for JSON Binding (JSR-367)
%Model View Controller (MVC) (JSR-371)
%Java API for WebSocket
%Java API for JSON Processing (JSON-P)
%Java API for RESTful Web Services (JAX-RS) JSR-370
%JavaServer Faces (JSF)
%Java Servlet JSR-369
%Expression Language (EL)
%Interceptors
%Java Message Service (JMS)
%Concurrency Utilities for Java EE
%Batch Applications for the Java Platform
%Contexts and Dependency Injection for Java EE (CDI) JSR-346
%Bean Validation
%Common Annotations
%Java Connector Architecture
%Java Transaction API (JTA)
%Java Persistence API (JPA) JSR-338
%Enterprise JavaBeans (EJB)
%JavaServer Pages (JSP)
%Web-Profile