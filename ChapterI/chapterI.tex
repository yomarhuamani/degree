\chapter{Problemática y Objetivos} \label{chapter:I}
%----------------------------------------Problematica---------------------------------------------------
\section{Problem\'{a}tica}
\begin{spacing}{1.5}	
	Tesla Technologies S.A.C, es una empresa dedicada a construir e implementar productos de software en materia de Riesgos, Seguridad de la información, Continuidad del negocio y Auditoría, con sede principal en el Perú y con la consigna de consolidarse como una opci\'{o}n s\'{o}lida en el mercado latinoamericano de productos digitales para el sector del gobierno corporativo de las empresas.\\	                         
	En el proceso org\'{a}nico de su crecimiento Tesla Technologies S.A.C  identifica un significativo incremento de nuevos requerimientos para crear reportes Adhoc por cada cliente y el manifiesto formal sobre las limitaciones de los reportes Adhoc en la elaboraci\'{o}n de sus informes gerenciales.\\
	El incremento de los requerimientos para crear nuevos reportes tienen valores significativos en los periodos 2016, 2017 y 2018 como muestra la Figura \ref{figure:chaperI_1}, generando un desequilibro en la carga operativa para el mantenimiento de reportes existentes y la atenci\'{o}n de nuevos requerimientos, impactando en el cumplimiento de los objetivos estrat\'{e}gicos de la empresa.\\
	Existieron intentos de darle dinamismo y flexibilidad a los reportes solicitados, sin embargo los entregables m\'{i}nimos no cuplieron con las espectativas de la gerencia y los clientes. La opci\'{o}n de desarrollar una nueva aplicaci\'{o}n para atender esta necesidad implicaba desatender y colocar en riesgo el cumplimiento de objetivos estrat\'{e}gicos, raz\'{o}n fundamental para pensar en una soluci\'{o}n diferente.\\
	Dando una mirada a proyectos digitales externos con funcionalidades para uns reporter\'{i}a din\'{a}mica, Tesla Technologies S.A.C toma la decisi\'{o}n de integrar sus productos con herramientas externas evaluando las ventajas y desventajas de esta apuesta. En este nuevo camino es el proyecto Saiku Analytics complemento de la suite Pentaho BI seleccionado para la integración con los productos de Tesla Technologies S.A.C .\\	
	Este proyecto de integraci\'{o}n si bien no abarca todo el proceso de evaluaci\'{o}n y selecci\'{o}n de  las herremientas de software del cuadrante de Gartner y del Wave Forrester para dar con la herramienta id\'{o}nea, enfoca su mirada en consideraciones de licencia y en los puntos tecnol\'{o}gicos importantes de compatibilidad y desarrollo de requisitos cr\'{i}ticos para lograr una integraci\'{o}n exitosa.\\
	
	\begin{figure}[H]
		\begin{center}
			\tikzI
		\end{center}
		\caption {\centering \small{Total de requisitos y requisitos para atender reportes,  2016-2018(\'{A}rea de soporte de Tesla Technologies S.A.C)}} \label{figure:chaperI_1}
	\end{figure}
	
\end{spacing}

\clearpage
%----------------------------------------Solucion Desarrolladora----------------------------------------
\section{Soluci\'{o}n desarrolladora}
\begin{spacing}{1.5}
	Realizar la integraci\'{o}n de funcionalidades del proyecto Open Source Saiku Analytics a la herramienta GiroSuite usando el marco de trabajo Scrum y la Programaci\'{o}n Extrema para la creaci\'{o}n din\'{a}mica de reportes.
\end{spacing}
%----------------------------------------Objectivo general----------------------------------------------
\section{Objetivo general}
\begin{spacing}{1.5}
	Realizar la integraci\'{o}n de funcionalidades del proyecto Open Source Saiku Analytics a la herramienta GiroSuite de la empresa Tesla Technologies S.A.C usando el marco de trabajo Scrum y la Programaci\'{o}n Extrema para la creaci\'{o}n din\'{a}mica de reportes.
\end{spacing}
%----------------------------------------Objetivos especificos------------------------------------------
\section{Objetivos espec\'{i}ficos}
\begin{spacing}{1.5}
	\begin{enumerate}[label=\alph*)]
		\item Disminuir el tiempo de los usuarios en la creaci\'{o}n de reportes gerenciales.
		\item Controlar la cantidad de reportes a medida que se desarrollan para la herramienta GiroSuite.
		\item Disminur el tiempo utilizado para el desarrollo de reportes a medida en la herramienta GiroSuite.
		\item Disminuir el costo operativo utilizado para el desarrollo de reportes a medida en la herramienta GiroSuite.
	\end{enumerate}	
\end{spacing}


	
