\chapter{Conclusiones y Recomendaciones} \label{chapter:IV}
\begin{spacing}{1.5}
\section{Conclusiones}
Lograr integrar las funcionalidades core del proyecto Saiku Analytics a la herramienta GiroSuite bajo el marco de trabajo Scrum conjuntamente con los valores, principios y pr\'{a}cticas de la Programaci\'{o}n Extrema(XP), fue un reto para el \textit{Scrum Team} y en particular para el \textit{Product Owner}, sin embargo, se logro alcanzar. La posibilidad de fracaso siempre estuvo presente y tener conciencia fue importante, las concluciones l\'{i}neas abajo detallan aspectos considerados y ejecutados por el \textit{Scrum Team} en el presente proyecto.
\begin{itemize}
	
	\item Se crearon los artefactos \textit{Product BackLog}, \textit{Sprint BackLog} y el \textit{Increment}.Los compromisos fueron muy importantes, durante todo el proyecto mantuvo presente en todo el \textit{Scrum Team} el valor aportado hacia los usuarios la integraci\'{o}n del proyecto Saiku Analytics a la herramiento GiroSuite.
	
	\item La planificaci\'{o}n y revisi\'{o}n de los Sprints aportaron un seguimiento detallado del avance del proyecto, adem\'{a} s de contribuir a identificar oportunidades de cambio y mejora. Las nuevas oportunidades de cambio y mejora retroalimentaron al \textit{Product BackLog} y el \textit{Scrum Team} propici\'{o} que sean antendidas en el siguiente \textit{Sprint Planing}.
	
	\item Nuestras actividades de Sprint Review y Sprint Restrospective demostraron aportar y afinar el cumplimiento de los compromisos en las posteriores iteraciones. Se comprob\'{o} mediante la experiencia de participar en las sesiones que  execeder el tiempo recomendado es en su mayoria suficiente para ejecutar estas actividades.
	
	\item Se realiz\'{o} cumplimiento del objetivo del proyecto con el aporte fundamental de la puesta en pr\'{a}ctica por el \textit{Scrum Team} de los valores recomendados por el \textit{Extreme Programming(XP)}. El \textit{Scrum Team} resalta el desaf\'{i}o de la ejecuci\'{o}n en el proyecto.
	
	\item Las pruebas unitarias en el desarrollo de las historias de usuario aportaron confianza a los \textit{Developers} en el desarrollo de pasar una tarea ``\textit{In Process}'' a ``Done''. El ambiente de integraci\'{o}n continua de la empresa Tesla Technologies S.A.C fue fundamental para las pruebas de integraci\'{o}.
	
	\item Las pruebas de aceptaci\'{o}n fue la confirmaci\'{o}n y el indicador del cumplimiento del compromiso del \textit{Product Backlog}, que fue afindandose en cada iteraci\'{o}n.
	
\end{itemize}
\clearpage
\section{Recomendaciones}
\end{spacing}