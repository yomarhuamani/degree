\clearpage
\phantomsection
\pdfbookmark[0]{Introducción}{introduccion}

\chapter*{Introducción}
\begin{spacing}{1.5}
	Una de las características importantes a considerar en un producto de software, es la flexibilidad o nivel de configuración del producto para responder a las existentes y nuevas necesidades de los usuarios. Aumentar el nivel de configuración de un producto de software tiene beneficios directos en el uso de recursos humanos al momento de atender las necesidades de los usuarios y la oportunidad de convertir una necesidad en una ventaja comparativa.\\
	En el presente informe revisamos la licencia, funcionalidades, tecnologías usadas y limitaciones del proyecto Saiku Analytics para la integración con los productos ofrecidos por Tesla Technologies con el objetivo principal de crear ventaja competitiva en sus productos.
	En el camino de alcanzar el objetivo principal también es importante continuar atendiendo las necesidades de los clientes, quienes presentan informes gerenciales con más frecuencia e intentan disminuir su tiempo en la construcción de estos informes.\\
	En el proceso de integración del proyecto Saiku Analytics con los productos que ofrece Tesla Technologies se consideran aspectos técnicos importantes de compatibilidad con el objetivo que el impacto positivo sea alto y el impacto negativo sea bajo o en lo posible no sea percibido por los usuarios. 
	Una de las características inherentes de todo producto son sus limitaciones, por tal motivo indicaremos algunas limitaciones temporales que se encontraron antes, durante y después de la integración del proyecto Saiku Analytics.\\
	Explorar nuevos proyectos de software privados y open source expande la visión y alcance de un producto de software, Realizar e invertir en esfuerzos que puedan estar al nivel de soluciones mundiales son el presagio de un futuro competitivo para el Perú.\\
\end{spacing}
