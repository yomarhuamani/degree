\clearpage
\pdfbookmark[0]{Introducción}{introduccion}
\begin{spacing}{1.5}
	\begin{LARGE}
			\LARGE {\textbf{\begin{center}
						Introducción
			\end{center}}}
	\end{LARGE}
	\vspace{1cm}
	Una de las características importantes a considerar en un producto de software, es la flexibilidad o nivel de configuración del producto para responder a las existentes y nuevas necesidades de los usuarios. Aumentar el nivel de configuración de un producto de software tiene beneficios directos en el uso de recursos al atender nuevas necesidades de los usuarios y la oportunidad de convertir una necesidad en una ventaja comparativa.\\
	El presente informe aborda el proceso de integración del proyecto Saiku Analytics con los productos que ofrece Tesla Technologies S.A.C y lograr el objetivo de atender las necesidades de los clientes, quienes presentan informes gerenciales con más frecuencia e intentan disminuir su tiempo en la construcción de estos informes teniendo en cuenta aspectos técnicos importantes de compatibilidad y prioridad de los requerimientos de integración con la consigna que el impacto positivo sea alto y el impacto negativo sea bajo o en lo posible pueda ser mitigado. Las consideraciones t\'{e}cnicas incluyen aspectos de tipo y versi\'{o}n de los lenguajes de programaci\'{o}n, frameworks de desarrollo frontend y backend, motores relacionales de bases de datos y motores OLAP.\\
	El primer capítulo describe la problemática y define el objetivo general y los objetivos específicos, delimitados por el marco teórico desarrollado en el segundo capítulo con conceptos y fundamentos sobre las licencias, tecnologías usadas y limitaciones del proyecto Saiku Analytics para la integración con los productos ofrecidos por Tesla Technologies S.A.C.\\
	En el tercer cap\'{i}tulo se desarrollan e implementan los requisitos de software para la integraci\'{o}n del proyecto Saiku Analytics usando el marco de trabajo Scrum junto a la Programaci\'{o}n Extrema(XP) como enfoque de desarrollo de software.\\
	En aras de lograr los objetivos propuestos encontramos lugares comunes e inherentes en la implementaci\'{o}n del proyecto revelando los beneficios, limitaciones, propuestas de mejora y nuevas lecciones aprendidas; plasmados en el cuarto capítulo donde indicamos nuestras conclusiones y recomendaciones obtenidas del presente proyecto.\\
\end{spacing}
